%-------------------------------------------------------------------------------
%	SECTION TITLE
%-------------------------------------------------------------------------------
\cvsection{Software}


%-------------------------------------------------------------------------------
%	CONTENT
%-------------------------------------------------------------------------------
\begin{cventries}

%---------------------------------------------------------
\cventry
    {Jonas Falck} % Job title
    {Conway`s Bubbletea} % Organization
    {github} % Location
    {2025} % Date(s)
    {
      \begin{cvitems} % Description(s) of tasks/responsibilities
        \item {Kleines Terminal User Interface(TUI) für meine Conway`s game of life implementierung in Go unter Zuhilfenahme der BubbleTea library.}
        \vspace{0.2cm}
        \newline\badge[awesome-skyblue]{Go} \badge[awesome-skyblue]{BubbleTea}
      \end{cvitems}
    }

    \cventry
    {Jonas Falck} % Job title
    {joe-nas.github.io} % Organization
    {\href{https://github.com/joe-nas/joe-nas.github.io}{github}} % Location
    {2024} % Date(s)
    {
      \begin{cvitems} % Description(s) of tasks/responsibilities
        \item {Meine persönliche nicht immer 100\% aktuelle Github Pages Seite.}
        \vspace{0.2cm}
        \newline\badge[awesome-skyblue]{javascript} \badge[awesome-skyblue]{React}
      \end{cvitems}
    }

  \cventry
    {Jonas Falck} % Job title
    {Iron Delirium - Workout‑Tracker} % Organization
    {\href{https://github.com/joe-nas/workout-app}{Backend@github},\href{https://github.com/joe-nas/workout-app-frontend-next}{Frontend@github}} % Location
    {2023 - fortfahrend} % Date(s)
    {
      \begin{cvitems} % Description(s) of tasks/responsibilities
        \item {Iron Delirium ist mein aktuelles Lern-projekt, ein Workout‑Tracker, der es Nutzern ermöglicht, ihre Workouts und Fortschritte zu tracken. Nutzer
        können ein Konto erstellen, sich anmelden/abmelden sowie Workouts erstellen. Diese können bearbeitet und gelöscht werden. Iron Delirium™
        ist mit Spring Boot/Security und MongoDB im Backend sowie Next.js/React und Tailwind im Frontend gebaut. Tests sind mit mit Hilfe JUnit,
        Mockito sowie Testcontainers geschrieben.}
        \vspace{0.2cm}
        \newline\badge[awesome-skyblue]{Java} \badge[awesome-skyblue]{Spring Boot} \badge[awesome-skyblue]{Spring Security} \badge[awesome-skyblue]{JWT}
        \badge[awesome-skyblue]{Testing} \badge[awesome-skyblue]{Testcontainers} \badge[awesome-skyblue]{javascript} \badge[awesome-skyblue]{Next.js} 
        \badge[awesome-skyblue]{React} \badge[awesome-skyblue]{MongoDB}
      \end{cvitems}
    }

%---------------------------------------------------------
  \cventry
    {Jonas Falck} % Job title
    {haploplotR - Visualisierung des Kopplungsungleichgewichts mithilfe von 1000-Genomes-Daten} % Organization
    {\href{https://github.com/joe-nas/haploplotR}{haploplotR@github}} % Location
    {2015} % Date(s)
    {
      \begin{cvitems} % Description(s) of tasks/responsibilities
        \item {HaploplotR - Tool zur Visualisierung des Kopplungsungleichgewichts in menschlichen Populationen basierend auf 1000-Genomes-Daten. 
        Erstellt LD-Plots zur Darstellung von Allelkorrelationen, die Forschenden helfen, relevante Genomregionen für Krankheiten zu identifizieren.}
        \vspace{0.2cm}
        \newline\badge[awesome-skyblue]{R} \badge[awesome-skyblue]{GGPLOT} \badge[awesome-skyblue]{grid} \badge[awesome-skyblue]{Visualisierung}
      \end{cvitems}
    }
%---------------------------------------------------------
\end{cventries}
