%!TEX TS-program = xelatex
%!TEX encoding = UTF-8 Unicode
% Awesome CV LaTeX Template for Cover Letter
%
% This template has been downloaded from:
% https://github.com/posquit0/Awesome-CV
%
% Authors:
% Claud D. Park <posquit0.bj@gmail.com>
% Lars Richter <mail@ayeks.de>
%
% Template license:
% CC BY-SA 4.0 (https://creativecommons.org/licenses/by-sa/4.0/)
%


%-------------------------------------------------------------------------------
% CONFIGURATIONS
%-------------------------------------------------------------------------------
% A4 paper size by default, use 'letterpaper' for US letter
\documentclass[11pt, a4paper]{awesome-cv}

% Configure page margins with geometry
\geometry{left=1.4cm, top=.8cm, right=1.4cm, bottom=1.8cm, footskip=.5cm}

% Color for highlights
% Awesome Colors: awesome-emerald, awesome-skyblue, awesome-red, awesome-pink, awesome-orange
%                 awesome-nephritis, awesome-concrete, awesome-darknight
\colorlet{awesome}{awesome-skyblue}
% Uncomment if you would like to specify your own color
% \definecolor{awesome}{HTML}{CA63A8}

% Colors for text
% Uncomment if you would like to specify your own color
% \definecolor{darktext}{HTML}{414141}
% \definecolor{text}{HTML}{333333}
% \definecolor{graytext}{HTML}{5D5D5D}
% \definecolor{lighttext}{HTML}{999999}
% \definecolor{sectiondivider}{HTML}{5D5D5D}

% Set false if you don't want to highlight section with awesome color
\setbool{acvSectionColorHighlight}{true}

% If you would like to change the social information separator from a pipe (|) to something else
\renewcommand{\acvHeaderSocialSep}{\quad\textbar\quad}


%-------------------------------------------------------------------------------
%	PERSONAL INFORMATION
%	Comment any of the lines below if they are not required
%-------------------------------------------------------------------------------
% Available options: circle|rectangle,edge/noedge,left/right
\photo[circle,noedge,left]{./examples/profile}
\name{Jonas}{Falck}
\position{Praktikant Fachinformatiker für Anwendungsentwicklung @ Cosinex}
\address{Holzstraße 1, 44869 Bochum, Deutschland}

\mobile{(+49) 1522 1529 541}
\email{jonas@falcken.de}
%\dateofbirth{January 1st, 1970}
\homepage{joe-nas.github.io}
\github{joe-nas}
% \linkedin{}
% \gitlab{gitlab-id}
% \stackoverflow{SO-id}{SO-name}
% \twitter{@twit}
% \skype{skype-id}
% \reddit{reddit-id}
% \medium{madium-id}
% \kaggle{kaggle-id}
% \hackerrank{hackerrank-id}
% \telegram{telegram-username}
% \googlescholar{googlescholar-id}{name-to-display}
%% \firstname and \lastname will be used
% \googlescholar{googlescholar-id}{}
% \extrainfo{extra information}

% \quote{``Be the change that you want to see in the world."}


%-------------------------------------------------------------------------------
%	LETTER INFORMATION
%	All of the below lines must be filled out
%-------------------------------------------------------------------------------
% The company being applied to
\vspace{-1cm}
\recipient
  {Personalabteilung @ Cosinex}
  {cosinex GmbH, Gesundheitscampus-Süd 31, 44801 Bochum}
% The date on the letter, default is the date of compilation
\letterdate{\today}
% The title of the letter
\lettertitle{Praktikant Fachinformatiker für Anwendungsentwicklung in Umschulung (IHK) @ Cosinex}
% How the letter is opened
\letteropening{Sehr geehrte Frau Bouhouch, sehr geehrte Frau Borghaus}
% How the letter is closed
\letterclosing{Mit freundlichen Grüßen,}
% Any enclosures with the letter
% \letterenclosure[Attached]{Curriculum Vitae}


%-------------------------------------------------------------------------------
\begin{document}
\vspace{-3mm}
% Print the header with above personal information
% Give optional argument to change alignment(C: center, L: left, R: right)
\makecvheader[R]
\vspace{-3mm}
% Print the footer with 3 arguments(<left>, <center>, <right>)
% Leave any of these blank if they are not needed
\makecvfooter
  {\today}
  {Jonas Falck~~~·~~~Anschreiben}
  {}

% Print the title with above letter information
\makelettertitle


%-------------------------------------------------------------------------------
%	LETTER CONTENT
%-------------------------------------------------------------------------------
\begin{cvletter}
  \vspace{-1.5mm}
  \lettersection{Über mich}
  Nach meinem Zivildienst habe ich an der Radboud Universität in Nimwegen, Niederlande, medizinische Biologie studiert. 
  Schon früh im Studium begeisterte ich mich für Bioinformatik, Datenanalyse und maschinelles Lernen und eignete mir diese Kenntnisse eigenständig an. 
  Parallel entwickelte ich Interesse an der Programmierung dynamischer Webseiten sowie der Server-Administration.
  Während meines Studiums arbeitete ich zunächst in Teilzeit, später in Vollzeit mehrere Jahre in der Logistik. 
  Obwohl ich den Teamzusammenhalt sehr schätzte, fehlte mir dort die geistige Herausforderung, sodass ich mich entschied, eine berufliche Neuausrichtung in die IT zu verfolgen.
  Daraufhin begann ich, mich mit Onlinekursen und eigenen Projekten auf den Quereinstieg in die IT vorzubereiten.
  Ende 2023 bin ich nach Deutschland zurückgekehrt und habe mich nach einer Phase der Jobsuche Mitte 2024 für eine Umschulung zum Fachinformatiker für Anwendungsentwicklung entschieden.
  Diese habe ich im März 2025 begonnen und fühle mich dort genau richtig aufgehoben.
  Die Umschulung bietet mir nicht nur die Chance, mein Fachwissen zu erweitern, sondern auch die Möglichkeit, anderen Mitschülern bei bestimmten Themen zur Seite zu stehen.
  Aktuell suche ich im Rahmen dieser Umschulung für den Zeitraum vom 19.01.2026 bis zum 19.10.2026 einen Praktikumsplatz - sehr gerne bei Cosinex.
  Weiterhin ist anzumerken, dass mir als Umschüler keine Ausbildungsvergütung zusteht und daher keine zusätzlichen Lohnkosten auf Sie zukommen.
  \vspace{-1.5mm}
  \lettersection{Warum Cosinex?}
  Während meiner Zeit in den Niederlanden habe ich aus Sicht eines Bürgers erlebt, wie unkompliziert der Staat digital mit seinen Bürgern kommuniziert, und die Vorteile einer funktionierenden digitalen Verwaltung schätzen gelernt.
  Bei Cosinex sehe ich die Chance, aktiv zur Digitalisierung in Deutschland beizutragen und gleichzeitig meine Expertise in Java und TypeScript zu vertiefen.
  Ich hoffe außerdem, bei Cosinex einen tieferen Einblick in den Software-Lifecycle zu erhalten und diesen aktiv mitgestalten zu können.
  Weiterhin reizt es mich, Software nicht nur als Einzelkämpfer, sondern im Team zu entwickeln.
  \vspace{-1.5mm}
  \lettersection{Warum ich?}
  Ich bin überzeugt, der richtige Kandidat für einen Praktikumsplatz bei Cosinex zu sein.
  Ich bringe ein großes Interesse an der Softwareentwicklung und IT mit, insbesondere an der Backend-Programmierung, da ich es faszinierend finde, komplexe und skalierbare Systeme zu entwickeln. Mit Java und TypeScript möchte ich meine Kenntnisse in diesem Bereich vertiefen und weiter ausbauen.
  Durch Projekte im Studium und private Vorhaben habe ich bereits Erfahrung in der Backend- und Frontend-Entwicklung, unter anderem mit Java, Spring, JavaScript und React, gesammelt.
  Meine größte Stärke ist meine ausgeprägte intrinsische Motivation und Neugierde.
  Beides ermöglicht es mir, mich mit Begeisterung in neue Technologien einzuarbeiten und kontinuierlich dazuzulernen.
  Ich bin überzeugt, dass ich als Praktikant einen wertvollen Beitrag zu Ihrem Team leisten kann, und stehe Ihnen gerne für ein persönliches Kennenlerngespräch zur Verfügung.
  \vspace{-1.5mm}
\end{cvletter}

\makeletterclosing
\end{document}
